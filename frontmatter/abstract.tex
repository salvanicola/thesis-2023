%!TEX root = ../dissertation.tex
% the abstract

With the increasing use of platforms on which an individual can make available his property for short-term rent, it grows the need of a smart way to manage the access rights to their holding. The common way for the guests to check-in or check-out during the accommodation involves physical key. For a host who lives or is located far away from the rented property, this can be time consuming and sometimes expensive to manage in person. Nowadays, many companies have developed various solutions for this problem, from key-pads, using temporary code numbers, which the guests must insert to have access grants, to properly door smart-locks, unlockable by simply approaching with the smartphone. This leads to the need to generate and manage secure virtual keys, which is a process that can even be automated, given the arrival and departure dates of the guests. 
\\ In this thesis, I describe my work at Kuama s.r.l., the company with which I designed and developed Kerbero. 
\\ Kerbero is an application that interfaces with smart-locks, in order to generate and manage secure virtual keys. It is designed to communicate with the external APIs of different smart-lock vendors, in order to retrieve and manage the devices available to the host. Moreover, it is able to generate and send temporary virtual keys to guests based on the reservation details provided. Kerbero is composed of a REST API and a Single-Page Application (SPA), which implements part of these features. \\ As such, in this document, I discuss in detail the analysis, the design, and the technical choices performed during this project.
