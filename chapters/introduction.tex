%!TEX root = ../dissertation.tex
\chapter{Introduction}
\label{introduction}

\newthought{The most annoying and time consuming thing} of renting a property for short term periods is managing the check-in and check-out of the guests. Airbnb hosts, for example, are compelled to meet their guests. In fact, they are obliged to deliver the keys and, at the end of the accommodation, they have to retrieve them. This turns out to be a problem when the property is located far from the owner, who has two alternatives: to hire a manager or to lose every time a huge amount of time reaching his rented house. 
\\ A solution for this problem comes from the \gls{IoT} world. In the last years is in development the technology of the smart-locks, which are devices that can replace the common door lock, in order to abandon the physical keys using. Smart-locks have different implementations from the simple keypads, to devices that can simply turn the key for you. The market of this technology is in constant growing, it is valued at USD 1.64 billion in 2021, with a growth rate of 19.5\% and reach in 2030 a value of USD 8.13 billion\cite{marketsize}.
\\ The advantages of not using physical keys are multiple: from security reason, thanks to the fact that you can not lost them, to the possibility of remotely manage everything, from the opening and closing process, to the selection of who is handling the access permissions. Moreover, if you are managing big properties with multiple entrances or areas, you can model a central system, which can control all the doors of the aforementioned building. 
\\ The use of the smart-locks in rented properties is constantly increasing, thanks to the spreading of platform like Airbnb. There are several reason, such as the possibility of reduce the managing time of the listed property, when the owners do not want to manage its houses full-time and even the security that this type of devices can provide. As such, they allow to lock and unlock their rental properties remotely using the smartphone, which, for example, can be especially useful in case of an emergency.
\\ There are also some potential drawbacks and challenges of using smart-locks in the Airbnb context. One concern is the upfront cost of purchasing and installing those devices, which can be significant for hosts. Additionally, there is the risk of technical issues or malfunctions, which could cause delays or inconvenience for guests. Finally, there is the potential for privacy concerns, since the smart-locks can record data about who has accessed a property and when.
\\ However, the monitoring feature is not always a drawback, since, from the hosts perspective, the smart-locks offer an additional security layer, especially for the owners which are not present at their properties during the entire duration of a guest's stay.
We will explore later that this trust balance between guests and host is important in platform like Airbnb, and how a smart-lock can have an impact on that.
\\ With the collaboration of Kuama s.r.l., I developed Kerbero, an application to manage and interface with multiple smart-locks and generate secured keys. Kerbero is developed modularly, since he can provide support to different smart-locks models and providers, through the simple adding of plugins. The first provider we started to supply the support, was Nuki. Nuki develops smart-locks to open and close the door, which are simple, secure and quite affordable. The peculiarity of this devices is that they do not need to replace the existing door system, since the smart-lock is designed to turn automatically the existing system. Nuki has several interfaces, but the common way is the Bluetooth using the proprietary application. Another way is through the web application, but the smart-lock must be connected to the internet via a dedicated bridge.
\\ Kerbero provides an host-oriented user interface, which allows the owners to connect their Nuki devices through the internet, linking their existing account. The application can generate temporary virtual keys which can be assigned to the upcoming guests. Kerbero is designed to be linked with vacation rental management services, like \gls{Airbnb} or \gls{Booking}, in order to retrieve the reservation information and automatically create the keys based on this data.
\\ In this thesis, we will explore the context in which the application is been conceived and the technologies involved in it, the analysis, the design and, finally the resulting evaluation of the project.
 
\section{Thesis Outline}

\begin{description}
    \item[{\hyperref[cap:one]{The second chapter}}] describes the context in which the project is developed and the scenario covered.
    
    \item[{\hyperref[cap:two]{The third chapter}}] is an insight on the smart-locks technologies, the interfaces and the security vulnerabilities.

    \item[{\hyperref[cap:three]{The fourth chapter}}] outline the Nuki solutions as a case study, with an insight on the component and framework with which their devices works.
    
    \item[{\hyperref[cap:four]{The fifth chapter}}] describes the analysis, the design and the implementation choices taken during the project Kerbero.

    \item[{\hyperref[cap:five]{The sixth chapter}}] is an evaluation on the choice done during the analysis and design, with a focus on the results and the workflow adopted.

    \item[{\hyperref[cap:conclusion]{The last chapter}}] is the conclusion of the thesis, in which I analyze the personal achievement and I expose a critical evaluation of my work.
\end{description}
